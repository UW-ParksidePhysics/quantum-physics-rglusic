%% Generated by Sphinx.
\def\sphinxdocclass{jupyterBook}
\documentclass[letterpaper,10pt,english]{jupyterBook}
\ifdefined\pdfpxdimen
   \let\sphinxpxdimen\pdfpxdimen\else\newdimen\sphinxpxdimen
\fi \sphinxpxdimen=.75bp\relax
%% turn off hyperref patch of \index as sphinx.xdy xindy module takes care of
%% suitable \hyperpage mark-up, working around hyperref-xindy incompatibility
\PassOptionsToPackage{hyperindex=false}{hyperref}
%% memoir class requires extra handling
\makeatletter\@ifclassloaded{memoir}
{\ifdefined\memhyperindexfalse\memhyperindexfalse\fi}{}\makeatother

\PassOptionsToPackage{warn}{textcomp}

\catcode`^^^^00a0\active\protected\def^^^^00a0{\leavevmode\nobreak\ }
\usepackage{cmap}
\usepackage{fontspec}
\defaultfontfeatures[\rmfamily,\sffamily,\ttfamily]{}
\usepackage{amsmath,amssymb,amstext}
\usepackage{polyglossia}
\setmainlanguage{english}



\setmainfont{FreeSerif}[
  Extension      = .otf,
  UprightFont    = *,
  ItalicFont     = *Italic,
  BoldFont       = *Bold,
  BoldItalicFont = *BoldItalic
]
\setsansfont{FreeSans}[
  Extension      = .otf,
  UprightFont    = *,
  ItalicFont     = *Oblique,
  BoldFont       = *Bold,
  BoldItalicFont = *BoldOblique,
]
\setmonofont{FreeMono}[
  Extension      = .otf,
  UprightFont    = *,
  ItalicFont     = *Oblique,
  BoldFont       = *Bold,
  BoldItalicFont = *BoldOblique,
]


\usepackage[Bjarne]{fncychap}
\usepackage[,numfigreset=1,mathnumfig]{sphinx}

\fvset{fontsize=\small}
\usepackage{geometry}


% Include hyperref last.
\usepackage{hyperref}
% Fix anchor placement for figures with captions.
\usepackage{hypcap}% it must be loaded after hyperref.
% Set up styles of URL: it should be placed after hyperref.
\urlstyle{same}


\usepackage{sphinxmessages}



         \usepackage[Latin,Greek]{ucharclasses}
        \usepackage{unicode-math}
        % fixing title of the toc
        \addto\captionsenglish{\renewcommand{\contentsname}{Contents}}
        

\title{Quantum Mechanics Problems}
\date{May 09, 2021}
\release{}
\author{Ryan J.\@{} Glusic}
\newcommand{\sphinxlogo}{\vbox{}}
\renewcommand{\releasename}{}
\makeindex
\begin{document}

\pagestyle{empty}
\sphinxmaketitle
\pagestyle{plain}
\sphinxtableofcontents
\pagestyle{normal}
\phantomsection\label{\detokenize{title-page::doc}}


\sphinxAtStartPar
This Jupyter book is an assortment of computational physics problems from within Griffith’s Quantum Mechanics, 3rd edition.


\chapter{Approximating the Probability Density for a Harmonic Oscillator}
\label{\detokenize{1.13:approximating-the-probability-density-for-a-harmonic-oscillator}}\label{\detokenize{1.13::doc}}
\sphinxAtStartPar
As per the problem, we need 1.11(b) solution to begin.

\sphinxAtStartPar
\(\rho(x)=\frac{1}{\pi\sqrt{a^2-x^2}}\), where \(a\) is our upper bound.

\sphinxAtStartPar
Integrating this, one can verify that it is indeed normalized and equal to one from some \(-a\) to \(a\).

\sphinxAtStartPar
This is our analytical solution, now our numerical solution is as follows:

\sphinxAtStartPar
\(\rho_{approx}(x)=\)a\(\cdot\cos(\omega t)\), where we let a and \(\omega\) both equal \(1\) as stated in the problem.

\begin{sphinxVerbatim}[commandchars=\\\{\}]
\PYG{k+kn}{import} \PYG{n+nn}{numpy} \PYG{k}{as} \PYG{n+nn}{np}
\PYG{k+kn}{import} \PYG{n+nn}{matplotlib}\PYG{n+nn}{.}\PYG{n+nn}{pyplot} \PYG{k}{as} \PYG{n+nn}{plt}

\PYG{c+c1}{\PYGZsh{} Constants}
\PYG{n}{a} \PYG{o}{=} \PYG{l+m+mi}{5}
\PYG{n}{w} \PYG{o}{=} \PYG{l+m+mi}{1}

\PYG{c+c1}{\PYGZsh{} Point generation}
\PYG{n}{x\PYGZus{}inputs} \PYG{o}{=} \PYG{n}{np}\PYG{o}{.}\PYG{n}{linspace}\PYG{p}{(}\PYG{o}{\PYGZhy{}}\PYG{n}{a}\PYG{p}{,} \PYG{n}{a}\PYG{p}{,} \PYG{l+m+mi}{10000}\PYG{p}{)}
\PYG{n}{analytical} \PYG{o}{=} \PYG{l+m+mi}{1}\PYG{o}{/}\PYG{p}{(}\PYG{n}{np}\PYG{o}{.}\PYG{n}{pi}\PYG{o}{*}\PYG{n}{np}\PYG{o}{.}\PYG{n}{sqrt}\PYG{p}{(}\PYG{n}{a}\PYG{o}{*}\PYG{o}{*}\PYG{l+m+mi}{2}\PYG{o}{\PYGZhy{}}\PYG{n}{x\PYGZus{}inputs}\PYG{o}{*}\PYG{o}{*}\PYG{l+m+mi}{2}\PYG{p}{)}\PYG{p}{)}
\PYG{n}{approx} \PYG{o}{=} \PYG{n}{a}\PYG{o}{*}\PYG{n}{np}\PYG{o}{.}\PYG{n}{cos}\PYG{p}{(}\PYG{n}{w}\PYG{o}{*}\PYG{n}{x\PYGZus{}inputs}\PYG{p}{)}
\end{sphinxVerbatim}

\begin{sphinxVerbatim}[commandchars=\\\{\}]
\PYGZlt{}ipython\PYGZhy{}input\PYGZhy{}1\PYGZhy{}0294d58c52ee\PYGZgt{}:10: RuntimeWarning: divide by zero encountered in true\PYGZus{}divide
  analytical = 1/(np.pi*np.sqrt(a**2\PYGZhy{}x\PYGZus{}inputs**2))
\end{sphinxVerbatim}

\begin{sphinxVerbatim}[commandchars=\\\{\}]
\PYG{n}{times} \PYG{o}{=} \PYG{n}{np}\PYG{o}{.}\PYG{n}{pi} \PYG{o}{*} \PYG{n}{np}\PYG{o}{.}\PYG{n}{random}\PYG{o}{.}\PYG{n}{rand}\PYG{p}{(}\PYG{l+m+mi}{10000}\PYG{p}{)}
\PYG{n}{cosine} \PYG{o}{=} \PYG{n}{np}\PYG{o}{.}\PYG{n}{cos}\PYG{p}{(}\PYG{n}{times}\PYG{p}{)}
\PYG{n}{plt}\PYG{o}{.}\PYG{n}{hist}\PYG{p}{(}\PYG{n}{cosine}\PYG{p}{)}
\PYG{n}{plt}\PYG{o}{.}\PYG{n}{plot}\PYG{p}{(}\PYG{n}{np}\PYG{o}{.}\PYG{n}{linspace}\PYG{p}{(}\PYG{o}{\PYGZhy{}}\PYG{l+m+mi}{1}\PYG{p}{,} \PYG{l+m+mi}{1}\PYG{p}{,} \PYG{l+m+mi}{10000}\PYG{p}{)}\PYG{p}{,} \PYG{l+m+mi}{2000}\PYG{o}{/}\PYG{p}{(}\PYG{n}{np}\PYG{o}{.}\PYG{n}{pi}\PYG{o}{*}\PYG{n}{np}\PYG{o}{.}\PYG{n}{sqrt}\PYG{p}{(}\PYG{l+m+mi}{1}\PYG{o}{*}\PYG{o}{*}\PYG{l+m+mi}{2} \PYG{o}{\PYGZhy{}} \PYG{n}{np}\PYG{o}{.}\PYG{n}{linspace}\PYG{p}{(}\PYG{o}{\PYGZhy{}}\PYG{l+m+mi}{1}\PYG{p}{,} \PYG{l+m+mi}{1}\PYG{p}{,} \PYG{l+m+mi}{10000}\PYG{p}{)}\PYG{o}{*}\PYG{o}{*}\PYG{l+m+mi}{2}\PYG{p}{)}\PYG{p}{)}\PYG{p}{)}
\PYG{n}{plt}\PYG{o}{.}\PYG{n}{ylim}\PYG{p}{(}\PYG{l+m+mi}{0}\PYG{p}{,} \PYG{l+m+mi}{2000}\PYG{p}{)}
\PYG{n}{plt}\PYG{o}{.}\PYG{n}{show}\PYG{p}{(}\PYG{p}{)}
\end{sphinxVerbatim}

\begin{sphinxVerbatim}[commandchars=\\\{\}]
\PYGZlt{}ipython\PYGZhy{}input\PYGZhy{}2\PYGZhy{}4a4cbf29a3b7\PYGZgt{}:4: RuntimeWarning: divide by zero encountered in true\PYGZus{}divide
  plt.plot(np.linspace(\PYGZhy{}1, 1, 10000), 2000/(np.pi*np.sqrt(1**2 \PYGZhy{} np.linspace(\PYGZhy{}1, 1, 10000)**2)))
\end{sphinxVerbatim}

\noindent\sphinxincludegraphics{{1.13_2_1}.png}


\chapter{Finding the Eigenstate Energies}
\label{\detokenize{2.56:finding-the-eigenstate-energies}}\label{\detokenize{2.56::doc}}
\sphinxAtStartPar
Problem 2.56 in Griffith’s Quantum Mechanics 3rd Edition asks us to use the ‘wag the dog’ method for finding \(E_1, E_2\) and \(E_3\) energy states for a quantum oscillator.


\chapter{Quantum Oscillator}
\label{\detokenize{2.56:quantum-oscillator}}\begin{equation*}
\begin{split}
\frac{\partial^2 \psi}{\partial \xi^2} = (\xi^2 - K)\psi
\end{split}
\end{equation*}\begin{equation*}
\begin{split}
\xi = \sqrt{\frac{m\omega}{\hbar}}x, K=\frac{2E}{\hbar\omega}
\end{split}
\end{equation*}

\chapter{Finding Even and Odd \protect\(E_n\protect\)}
\label{\detokenize{2.56:finding-even-and-odd-e-n}}
\sphinxAtStartPar
As hinted at in the problem description \sphinxcode{\sphinxupquote{For the first (and third) excited state you will need to set}} \(\psi(\xi)=0, \psi'(\xi)=1\). We need to flip our initial conditions to find the first and third energy states. That is for \(n=1\) and \(n=3\), odd \(n\) correspond to the odd eigenstates of the wave function; even \(n\) correspond to even eigenstates.


\chapter{Wag the Dog}
\label{\detokenize{2.56:wag-the-dog}}
\sphinxAtStartPar
The following code will numerically solve Eq. (1) using the wag the dog method. It’s also listed in the common folder in this repository.

\begin{sphinxVerbatim}[commandchars=\\\{\}]
\PYG{k+kn}{import} \PYG{n+nn}{numpy} \PYG{k}{as} \PYG{n+nn}{np}
\PYG{k+kn}{from} \PYG{n+nn}{scipy}\PYG{n+nn}{.}\PYG{n+nn}{integrate} \PYG{k+kn}{import} \PYG{n}{solve\PYGZus{}ivp}
\PYG{k+kn}{import} \PYG{n+nn}{matplotlib}\PYG{n+nn}{.}\PYG{n+nn}{pyplot} \PYG{k}{as} \PYG{n+nn}{plt}

\PYG{k}{def} \PYG{n+nf}{wag\PYGZus{}the\PYGZus{}dog}\PYG{p}{(}\PYG{n}{x\PYGZus{}range}\PYG{o}{=}\PYG{p}{(}\PYG{l+m+mi}{0}\PYG{p}{,}\PYG{l+m+mi}{5}\PYG{p}{)}\PYG{p}{,} \PYG{n}{k\PYGZus{}values}\PYG{o}{=}\PYG{n}{np}\PYG{o}{.}\PYG{n}{linspace}\PYG{p}{(}\PYG{l+m+mf}{0.9}\PYG{p}{,} \PYG{l+m+mf}{1.1}\PYG{p}{,} \PYG{l+m+mi}{10}\PYG{p}{)}\PYG{p}{,} \PYG{n}{initial\PYGZus{}values}\PYG{o}{=}\PYG{p}{(}\PYG{l+m+mf}{1.}\PYG{p}{,}\PYG{l+m+mf}{0.}\PYG{p}{)}\PYG{p}{)}\PYG{p}{:}
    \PYG{l+s+sd}{\PYGZdq{}\PYGZdq{}\PYGZdq{}}
\PYG{l+s+sd}{        This function numerically solves the wave equation for a harmonic oscillator, based on a \PYGZsq{}guess\PYGZsq{} for k and goes from there. This is for the problem 2.55 in Griffth\PYGZsq{}s quantum mechanics (3rd edition).}
\PYG{l+s+sd}{    \PYGZdq{}\PYGZdq{}\PYGZdq{}}
    \PYG{n}{positions} \PYG{o}{=} \PYG{n}{np}\PYG{o}{.}\PYG{n}{linspace}\PYG{p}{(}\PYG{n}{x\PYGZus{}range}\PYG{p}{[}\PYG{l+m+mi}{0}\PYG{p}{]}\PYG{p}{,} \PYG{n}{x\PYGZus{}range}\PYG{p}{[}\PYG{l+m+mi}{1}\PYG{p}{]}\PYG{p}{,} \PYG{l+m+mi}{1000}\PYG{p}{)}
    \PYG{n}{plt}\PYG{o}{.}\PYG{n}{figure}\PYG{p}{(}\PYG{n}{figsize}\PYG{o}{=}\PYG{p}{(}\PYG{l+m+mi}{24}\PYG{p}{,}\PYG{l+m+mi}{8}\PYG{p}{)}\PYG{p}{)}

    \PYG{n}{allowed\PYGZus{}potentials} \PYG{o}{=} \PYG{p}{[}\PYG{l+s+s1}{\PYGZsq{}}\PYG{l+s+s1}{harmonic oscillator}\PYG{l+s+s1}{\PYGZsq{}}\PYG{p}{]}
    \PYG{k}{for} \PYG{n}{k\PYGZus{}value} \PYG{o+ow}{in} \PYG{n}{k\PYGZus{}values}\PYG{p}{:}  
        \PYG{c+c1}{\PYGZsh{} Differential equation}
        \PYG{n}{psi\PYGZus{}prime} \PYG{o}{=} \PYG{k}{lambda} \PYG{n}{xi}\PYG{p}{,} \PYG{n}{psi}\PYG{p}{,} \PYG{n}{k}\PYG{p}{:} \PYG{p}{[}\PYG{n}{psi}\PYG{p}{[}\PYG{l+m+mi}{1}\PYG{p}{]}\PYG{p}{,} \PYG{p}{(}\PYG{n}{xi}\PYG{o}{*}\PYG{o}{*}\PYG{l+m+mi}{2}\PYG{o}{\PYGZhy{}}\PYG{n}{k}\PYG{p}{)}\PYG{o}{*}\PYG{n}{psi}\PYG{p}{[}\PYG{l+m+mi}{0}\PYG{p}{]}\PYG{p}{]}

        \PYG{c+c1}{\PYGZsh{} Solve differential equation using scipy}
        \PYG{n}{sol} \PYG{o}{=} \PYG{n}{solve\PYGZus{}ivp}\PYG{p}{(}\PYG{n}{psi\PYGZus{}prime}\PYG{p}{,} \PYG{n}{x\PYGZus{}range}\PYG{p}{,} \PYG{n}{initial\PYGZus{}values}\PYG{p}{,} \PYG{n}{t\PYGZus{}eval}\PYG{o}{=}\PYG{n}{positions}\PYG{p}{,} \PYG{n}{args}\PYG{o}{=}\PYG{p}{[}\PYG{n}{k\PYGZus{}value}\PYG{p}{]}\PYG{p}{)}

        \PYG{c+c1}{\PYGZsh{} Plot solution}
        \PYG{n}{plt}\PYG{o}{.}\PYG{n}{plot}\PYG{p}{(}\PYG{n}{positions}\PYG{p}{,} \PYG{n}{sol}\PYG{o}{.}\PYG{n}{y}\PYG{p}{[}\PYG{l+m+mi}{0}\PYG{p}{]}\PYG{p}{,} \PYG{n}{label}\PYG{o}{=}\PYG{l+s+sa}{fr}\PYG{l+s+s1}{\PYGZsq{}}\PYG{l+s+s1}{\PYGZdl{}k\PYGZdl{}: }\PYG{l+s+si}{\PYGZob{}}\PYG{n}{k\PYGZus{}value}\PYG{l+s+si}{:}\PYG{l+s+s1}{.4f}\PYG{l+s+si}{\PYGZcb{}}\PYG{l+s+s1}{\PYGZsq{}}\PYG{p}{)}

    \PYG{c+c1}{\PYGZsh{} Plotting configuration}
    \PYG{n}{plt}\PYG{o}{.}\PYG{n}{legend}\PYG{p}{(}\PYG{p}{)}
    \PYG{n}{plt}\PYG{o}{.}\PYG{n}{axhline}\PYG{p}{(}\PYG{n}{c}\PYG{o}{=}\PYG{l+s+s1}{\PYGZsq{}}\PYG{l+s+s1}{black}\PYG{l+s+s1}{\PYGZsq{}}\PYG{p}{)}
    \PYG{n}{plt}\PYG{o}{.}\PYG{n}{xlabel}\PYG{p}{(}\PYG{l+s+sa}{r}\PYG{l+s+s1}{\PYGZsq{}}\PYG{l+s+s1}{\PYGZdl{}}\PYG{l+s+s1}{\PYGZbs{}}\PYG{l+s+s1}{xi\PYGZdl{}}\PYG{l+s+s1}{\PYGZsq{}}\PYG{p}{)}
    \PYG{n}{plt}\PYG{o}{.}\PYG{n}{ylabel}\PYG{p}{(}\PYG{l+s+sa}{r}\PYG{l+s+s1}{\PYGZsq{}}\PYG{l+s+s1}{\PYGZdl{}}\PYG{l+s+s1}{\PYGZbs{}}\PYG{l+s+s1}{psi\PYGZus{}n(}\PYG{l+s+s1}{\PYGZbs{}}\PYG{l+s+s1}{xi)\PYGZdl{}}\PYG{l+s+s1}{\PYGZsq{}}\PYG{p}{)}
    \PYG{n}{plt}\PYG{o}{.}\PYG{n}{ylim}\PYG{p}{(}\PYG{o}{\PYGZhy{}}\PYG{l+m+mi}{1}\PYG{p}{,}\PYG{l+m+mi}{1}\PYG{p}{)}
    \PYG{n}{plt}\PYG{o}{.}\PYG{n}{show}\PYG{p}{(}\PYG{p}{)}
\end{sphinxVerbatim}


\chapter{Using the Above Code}
\label{\detokenize{2.56:using-the-above-code}}
\sphinxAtStartPar
Using the above code, we can specify different K values to find the eigenstates through trial and error. Let’s start by finding the even eigenstates.

\begin{sphinxVerbatim}[commandchars=\\\{\}]
\PYG{n}{wag\PYGZus{}the\PYGZus{}dog}\PYG{p}{(}\PYG{n}{x\PYGZus{}range}\PYG{o}{=}\PYG{p}{(}\PYG{l+m+mi}{0}\PYG{p}{,}\PYG{l+m+mi}{10}\PYG{p}{)}\PYG{p}{,} \PYG{n}{k\PYGZus{}values}\PYG{o}{=}\PYG{p}{(}\PYG{l+m+mf}{2.9993}\PYG{p}{,} \PYG{l+m+mf}{3.0001}\PYG{p}{,} \PYG{l+m+mf}{6.9900}\PYG{p}{,} \PYG{l+m+mf}{7.0001}\PYG{p}{)}\PYG{p}{,} \PYG{n}{initial\PYGZus{}values}\PYG{o}{=}\PYG{p}{(}\PYG{l+m+mf}{0.}\PYG{p}{,}\PYG{l+m+mf}{1.}\PYG{p}{)}\PYG{p}{)}
\end{sphinxVerbatim}

\noindent\sphinxincludegraphics{{2.56_3_0}.png}


\chapter{Calculating the Odd Eigenstate Energies}
\label{\detokenize{2.56:calculating-the-odd-eigenstate-energies}}
\sphinxAtStartPar
As seen in the above code, it appears that around \(K=3\) and \(K=7\) the tail end of the wave function flips about the \(\xi\) axis. This indicates that the two lowest odd energy states \(E_1, E_3\) are here as the graph did not flip anywhere below these points. We calculate the energy values as follows:
\begin{equation*}
\begin{split}
K=\frac{2E_1}{\hbar\omega} \approx 3 \implies E_1 = \frac{3}{2}\hbar\omega
\end{split}
\end{equation*}\begin{equation*}
\begin{split}
K=\frac{2E_3}{\hbar\omega} \approx 7 \implies E_3 = \frac{7}{2}\hbar\omega
\end{split}
\end{equation*}

\chapter{Calculating the Even Eigenstate Energies}
\label{\detokenize{2.56:calculating-the-even-eigenstate-energies}}
\sphinxAtStartPar
Using the new boundary conditions as stated above, we can calculate the energies for the even states (\(E_2\))

\begin{sphinxVerbatim}[commandchars=\\\{\}]
\PYG{n}{wag\PYGZus{}the\PYGZus{}dog}\PYG{p}{(}\PYG{n}{x\PYGZus{}range}\PYG{o}{=}\PYG{p}{(}\PYG{l+m+mi}{0}\PYG{p}{,}\PYG{l+m+mi}{10}\PYG{p}{)}\PYG{p}{,} \PYG{n}{k\PYGZus{}values}\PYG{o}{=}\PYG{p}{(}\PYG{l+m+mf}{4.999}\PYG{p}{,}\PYG{l+m+mf}{5.001}\PYG{p}{)}\PYG{p}{,} \PYG{n}{initial\PYGZus{}values}\PYG{o}{=}\PYG{p}{(}\PYG{l+m+mf}{1.}\PYG{p}{,}\PYG{l+m+mf}{0.}\PYG{p}{)}\PYG{p}{)}
\end{sphinxVerbatim}

\noindent\sphinxincludegraphics{{2.56_6_0}.png}

\sphinxAtStartPar
As we did above for the odd states,
\begin{equation*}
\begin{split}
K=\frac{2E_2}{\hbar\omega} \approx 5 \implies E_2 = \frac{5}{2}\hbar\omega
\end{split}
\end{equation*}

\chapter{Problem 2.61}
\label{\detokenize{2.61:problem-2-61}}\label{\detokenize{2.61::doc}}
\sphinxAtStartPar
One way to obtain the allowed energies of a potential well numerically is to turn the Schrödinger equation into a matrix equation, by discretizing the variable x. Slice the relevant interval at evenly spaced points.
\begin{equation*}
\begin{split}
-\lambda \psi_{j-1} + (2\lambda + V_j)\psi - \lambda \psi_{j-1} = E\psi_j, \lambda = \frac{\hbar^2}{2m(\Delta x)^2}
\end{split}
\end{equation*}
\begin{sphinxVerbatim}[commandchars=\\\{\}]
\PYG{c+c1}{\PYGZsh{} Create Hamiltonian Matrix}
\PYG{k+kn}{import} \PYG{n+nn}{numpy} \PYG{k}{as} \PYG{n+nn}{np}
\PYG{k}{def} \PYG{n+nf}{create\PYGZus{}hamiltonian}\PYG{p}{(}\PYG{n}{dimension}\PYG{p}{,} \PYG{n}{end\PYGZus{}point}\PYG{o}{=}\PYG{l+m+mi}{0}\PYG{p}{)}\PYG{p}{:}
    \PYG{n}{x\PYGZus{}naught} \PYG{o}{=} \PYG{l+m+mf}{0.}
    \PYG{n}{x\PYGZus{}end}    \PYG{o}{=} \PYG{n}{end\PYGZus{}point}
    \PYG{n}{delta\PYGZus{}x}  \PYG{o}{=} \PYG{n}{end\PYGZus{}point}\PYG{o}{/}\PYG{p}{(}\PYG{n}{dimension}\PYG{o}{+}\PYG{l+m+mi}{1}\PYG{p}{)}
    \PYG{n}{potential\PYGZus{}well} \PYG{o}{=} \PYG{l+m+mf}{0.} \PYG{c+c1}{\PYGZsh{} Always zero inside the well}

    \PYG{c+c1}{\PYGZsh{} Let hbar = 1, m = 1.}
    \PYG{n}{ham\PYGZus{}mat} \PYG{o}{=} \PYG{n}{np}\PYG{o}{.}\PYG{n}{zeros}\PYG{p}{(}\PYG{p}{(}\PYG{n}{dimension}\PYG{p}{,}\PYG{n}{dimension}\PYG{p}{)}\PYG{p}{,} \PYG{n+nb}{float}\PYG{p}{)}
    \PYG{n}{np}\PYG{o}{.}\PYG{n}{fill\PYGZus{}diagonal}\PYG{p}{(}\PYG{n}{ham\PYGZus{}mat}\PYG{p}{[}\PYG{p}{:}\PYG{p}{,}\PYG{l+m+mi}{1}\PYG{p}{:}\PYG{p}{]}\PYG{p}{,} \PYG{o}{\PYGZhy{}}\PYG{l+m+mi}{1}\PYG{p}{)}
    \PYG{n}{np}\PYG{o}{.}\PYG{n}{fill\PYGZus{}diagonal}\PYG{p}{(}\PYG{n}{ham\PYGZus{}mat}\PYG{p}{,} \PYG{l+m+mi}{2}\PYG{p}{)}
    \PYG{n}{np}\PYG{o}{.}\PYG{n}{fill\PYGZus{}diagonal}\PYG{p}{(}\PYG{n}{ham\PYGZus{}mat}\PYG{p}{[}\PYG{l+m+mi}{1}\PYG{p}{:}\PYG{p}{]}\PYG{p}{,} \PYG{o}{\PYGZhy{}}\PYG{l+m+mi}{1}\PYG{p}{)}
    \PYG{n}{w}\PYG{p}{,} \PYG{n}{v} \PYG{o}{=} \PYG{n}{np}\PYG{o}{.}\PYG{n}{linalg}\PYG{o}{.}\PYG{n}{eig}\PYG{p}{(}\PYG{n}{ham\PYGZus{}mat}\PYG{p}{)}
    \PYG{k}{return} \PYG{p}{(}\PYG{n}{w}\PYG{p}{,}\PYG{n}{v}\PYG{p}{)}
\PYG{n}{create\PYGZus{}hamiltonian}\PYG{p}{(}\PYG{l+m+mi}{100}\PYG{p}{)}
\end{sphinxVerbatim}

\begin{sphinxVerbatim}[commandchars=\\\{\}]
(array([3.99903256e+00, 3.99613119e+00, 3.99129870e+00, 3.98453974e+00,
        3.97586088e+00, 3.96527050e+00, 3.95277884e+00, 9.67435416e\PYGZhy{}04,
        3.86880573e\PYGZhy{}03, 8.70130406e\PYGZhy{}03, 1.90671922e+00, 1.84463231e+00,
        1.96889638e+00, 2.03110362e+00, 2.09328078e+00, 3.93839800e+00,
        2.15536769e+00, 1.78269570e+00, 1.54602553e\PYGZhy{}02, 2.21730430e+00,
        2.40161415e+00, 2.34048711e+00, 1.65951289e+00, 1.59838585e+00,
        1.53764736e+00, 2.41391205e\PYGZhy{}02, 2.46235264e+00, 3.90402622e+00,
        3.88406853e+00, 1.47735615e+00, 2.52264385e+00, 3.86228812e+00,
        3.47295036e\PYGZhy{}02, 1.41757058e+00, 3.83870608e+00, 4.72211589e\PYGZhy{}02,
        2.58242942e+00, 1.35834846e+00, 6.16020016e\PYGZhy{}02, 1.29974710e+00,
        3.81334520e+00, 2.64165154e+00, 3.78623003e+00, 1.24182319e+00,
        2.70025290e+00, 1.15931473e\PYGZhy{}01, 9.59737849e\PYGZhy{}02, 1.37711876e\PYGZhy{}01,
        2.75817681e+00, 3.75738680e+00, 1.61293922e\PYGZhy{}01, 1.07267294e+00,
        9.64300750e\PYGZhy{}01, 1.01801184e+00, 9.11591634e\PYGZhy{}01, 1.86654798e\PYGZhy{}01,
        2.87176884e+00, 2.92732706e+00, 3.08840837e+00, 3.14006452e+00,
        3.19061773e+00, 3.03569925e+00, 8.59935484e\PYGZhy{}01, 2.13769968e\PYGZhy{}01,
        3.42516793e+00, 3.24001909e+00, 3.46811706e+00, 8.09382271e\PYGZhy{}01,
        3.38084004e+00, 2.42613200e\PYGZhy{}01, 3.50964588e+00, 7.59980905e\PYGZhy{}01,
        5.31882942e\PYGZhy{}01, 5.74832072e\PYGZhy{}01, 4.90354122e\PYGZhy{}01, 3.33517628e+00,
        2.98198816e+00, 6.19159959e\PYGZhy{}01, 4.50285786e\PYGZhy{}01, 2.73156590e\PYGZhy{}01,
        3.72684341e+00, 3.54971421e+00, 4.11716698e\PYGZhy{}01, 3.58828330e+00,
        6.64823720e\PYGZhy{}01, 3.62531583e+00, 3.69462941e+00, 3.74684172e\PYGZhy{}01,
        1.12823116e+00, 3.92214188e+00, 3.05370590e\PYGZhy{}01, 3.66077597e+00,
        3.39224034e\PYGZhy{}01, 7.78581192e\PYGZhy{}02, 3.28822082e+00, 7.11779177e\PYGZhy{}01,
        1.18463277e+00, 2.81536723e+00, 1.72096932e+00, 2.27903068e+00]),
 array([[\PYGZhy{}0.00437636,  0.00874848, \PYGZhy{}0.01311214, ...,  0.12849426,
          0.13934326, \PYGZhy{}0.13934326],
        [ 0.00874848, \PYGZhy{}0.01746312,  0.02611019, ..., \PYGZhy{}0.10477001,
          0.03888104,  0.03888104],
        [\PYGZhy{}0.01311214,  0.02611019, \PYGZhy{}0.03888104, ..., \PYGZhy{}0.04306823,
         \PYGZhy{}0.12849426,  0.12849426],
        ...,
        [ 0.01311214,  0.02611019,  0.03888104, ...,  0.04306823,
          0.12849426,  0.12849426],
        [\PYGZhy{}0.00874848, \PYGZhy{}0.01746312, \PYGZhy{}0.02611019, ...,  0.10477001,
         \PYGZhy{}0.03888104,  0.03888104],
        [ 0.00437636,  0.00874848,  0.01311214, ..., \PYGZhy{}0.12849426,
         \PYGZhy{}0.13934326, \PYGZhy{}0.13934326]]))
\end{sphinxVerbatim}


\chapter{Problem 2.62}
\label{\detokenize{2.62-harmonic:problem-2-62}}\label{\detokenize{2.62-harmonic::doc}}
\sphinxAtStartPar
Similar to problem 2.62, except now the potential well is \(V(x)=\frac{1}{2}kx^2\) (harmonic oscillator). Lets make our life simple and do as the book said: (\(\hbar=1, m=1, a=1\)).


\chapter{Generating the Hamilitonian}
\label{\detokenize{2.62-harmonic:generating-the-hamilitonian}}
\sphinxAtStartPar
The following code will generate the Hamilitonian matrix for our problem.

\begin{sphinxVerbatim}[commandchars=\\\{\}]
\PYG{c+c1}{\PYGZsh{} Create Hamiltonian Matrix}
\PYG{k+kn}{import} \PYG{n+nn}{numpy} \PYG{k}{as} \PYG{n+nn}{np}
\PYG{k+kn}{import} \PYG{n+nn}{scipy}\PYG{n+nn}{.}\PYG{n+nn}{sparse}\PYG{n+nn}{.}\PYG{n+nn}{linalg} \PYG{k}{as} \PYG{n+nn}{sp}

\PYG{k}{def} \PYG{n+nf}{create\PYGZus{}hamiltonian}\PYG{p}{(}\PYG{n}{dimension}\PYG{p}{,} \PYG{n}{end\PYGZus{}point}\PYG{o}{=}\PYG{l+m+mi}{0}\PYG{p}{)}\PYG{p}{:}
    \PYG{c+c1}{\PYGZsh{} Let hbar = 1, m = 1.}
    \PYG{n}{ham\PYGZus{}mat} \PYG{o}{=} \PYG{n}{np}\PYG{o}{.}\PYG{n}{zeros}\PYG{p}{(}\PYG{p}{(}\PYG{n}{dimension}\PYG{p}{,}\PYG{n}{dimension}\PYG{p}{)}\PYG{p}{,} \PYG{n+nb}{float}\PYG{p}{)}

    \PYG{c+c1}{\PYGZsh{} Fill upper diagonal}
    \PYG{n}{np}\PYG{o}{.}\PYG{n}{fill\PYGZus{}diagonal}\PYG{p}{(}\PYG{n}{ham\PYGZus{}mat}\PYG{p}{[}\PYG{p}{:}\PYG{p}{,}\PYG{l+m+mi}{1}\PYG{p}{:}\PYG{p}{]}\PYG{p}{,} \PYG{o}{\PYGZhy{}}\PYG{l+m+mi}{1}\PYG{p}{)}

    \PYG{c+c1}{\PYGZsh{} Fill diagonal (potential changes the main diagonal)}
    \PYG{n}{diag} \PYG{o}{=} \PYG{p}{[}\PYG{p}{(}\PYG{l+m+mi}{2} \PYG{o}{+} \PYG{l+m+mi}{1}\PYG{o}{/}\PYG{l+m+mi}{2} \PYG{o}{*} \PYG{p}{(}\PYG{n}{j}\PYG{o}{/}\PYG{n}{dimension}\PYG{p}{)}\PYG{o}{*}\PYG{o}{*}\PYG{l+m+mi}{2}\PYG{p}{)} \PYG{k}{for} \PYG{n}{j} \PYG{o+ow}{in} \PYG{n+nb}{range}\PYG{p}{(}\PYG{l+m+mi}{1}\PYG{p}{,}\PYG{n}{dimension}\PYG{o}{+}\PYG{l+m+mi}{1}\PYG{p}{)}\PYG{p}{]}
    \PYG{n}{np}\PYG{o}{.}\PYG{n}{fill\PYGZus{}diagonal}\PYG{p}{(}\PYG{n}{ham\PYGZus{}mat}\PYG{p}{,} \PYG{n}{diag}\PYG{p}{)}

    \PYG{c+c1}{\PYGZsh{} Fill lower diagonal}
    \PYG{n}{np}\PYG{o}{.}\PYG{n}{fill\PYGZus{}diagonal}\PYG{p}{(}\PYG{n}{ham\PYGZus{}mat}\PYG{p}{[}\PYG{l+m+mi}{1}\PYG{p}{:}\PYG{p}{]}\PYG{p}{,} \PYG{o}{\PYGZhy{}}\PYG{l+m+mi}{1}\PYG{p}{)}

    \PYG{c+c1}{\PYGZsh{} Retrieve vectors and values then return them}
    \PYG{n}{w}\PYG{p}{,} \PYG{n}{v} \PYG{o}{=} \PYG{n}{sp}\PYG{o}{.}\PYG{n}{eigs}\PYG{p}{(}\PYG{n}{ham\PYGZus{}mat}\PYG{p}{,} \PYG{n}{k}\PYG{o}{=}\PYG{l+m+mi}{3}\PYG{p}{,} \PYG{n}{which}\PYG{o}{=}\PYG{l+s+s1}{\PYGZsq{}}\PYG{l+s+s1}{SR}\PYG{l+s+s1}{\PYGZsq{}}\PYG{p}{,} \PYG{n}{return\PYGZus{}eigenvectors}\PYG{o}{=}\PYG{k+kc}{True}\PYG{p}{)}
    \PYG{k}{return} \PYG{p}{(}\PYG{n}{w}\PYG{p}{,}\PYG{n}{v}\PYG{p}{)}
\end{sphinxVerbatim}


\chapter{Using the Above Hamilitonian}
\label{\detokenize{2.62-harmonic:using-the-above-hamilitonian}}
\sphinxAtStartPar
Lets use our function above to generate a 100x100 matrix and find the first three energy states along with the wave function values for these three energy states. We can then see the approximate wave function by plotting the results.

\begin{sphinxVerbatim}[commandchars=\\\{\}]
\PYG{k+kn}{import} \PYG{n+nn}{matplotlib}\PYG{n+nn}{.}\PYG{n+nn}{pyplot} \PYG{k}{as} \PYG{n+nn}{plt}
\PYG{n}{evals}\PYG{p}{,} \PYG{n}{evecs} \PYG{o}{=} \PYG{n}{create\PYGZus{}hamiltonian}\PYG{p}{(}\PYG{l+m+mi}{100}\PYG{p}{)}

\PYG{n}{plt}\PYG{o}{.}\PYG{n}{figure}\PYG{p}{(}\PYG{n}{figsize}\PYG{o}{=}\PYG{p}{(}\PYG{l+m+mi}{24}\PYG{p}{,}\PYG{l+m+mi}{8}\PYG{p}{)}\PYG{p}{)}

\PYG{c+c1}{\PYGZsh{} Plot each eigenstate}
\PYG{k}{for} \PYG{n}{n} \PYG{o+ow}{in} \PYG{n+nb}{range}\PYG{p}{(}\PYG{l+m+mi}{1}\PYG{p}{,}\PYG{l+m+mi}{4}\PYG{p}{)}\PYG{p}{:}
    \PYG{n}{plt}\PYG{o}{.}\PYG{n}{plot}\PYG{p}{(}\PYG{n}{evecs}\PYG{o}{.}\PYG{n}{real}\PYG{p}{[}\PYG{p}{:}\PYG{p}{,} \PYG{n}{n}\PYG{o}{\PYGZhy{}}\PYG{l+m+mi}{1}\PYG{p}{]}\PYG{p}{,} \PYG{n}{label}\PYG{o}{=}\PYG{l+s+sa}{fr}\PYG{l+s+s1}{\PYGZsq{}}\PYG{l+s+s1}{Energy \PYGZdl{}E\PYGZus{}}\PYG{l+s+si}{\PYGZob{}}\PYG{n}{n}\PYG{l+s+si}{\PYGZcb{}}\PYG{l+s+s1}{\PYGZdl{}: }\PYG{l+s+si}{\PYGZob{}}\PYG{l+m+mi}{100}\PYG{o}{*}\PYG{o}{*}\PYG{l+m+mi}{2}\PYG{o}{*}\PYG{n}{evals}\PYG{o}{.}\PYG{n}{real}\PYG{p}{[}\PYG{n}{n}\PYG{o}{\PYGZhy{}}\PYG{l+m+mi}{1}\PYG{p}{]}\PYG{l+s+si}{:}\PYG{l+s+s1}{.2f}\PYG{l+s+si}{\PYGZcb{}}\PYG{l+s+s1}{\PYGZsq{}}\PYG{p}{)}

\PYG{n}{plt}\PYG{o}{.}\PYG{n}{xlabel}\PYG{p}{(}\PYG{l+s+sa}{r}\PYG{l+s+s1}{\PYGZsq{}}\PYG{l+s+s1}{\PYGZdl{}x\PYGZdl{}}\PYG{l+s+s1}{\PYGZsq{}}\PYG{p}{)}
\PYG{n}{plt}\PYG{o}{.}\PYG{n}{ylabel}\PYG{p}{(}\PYG{l+s+sa}{r}\PYG{l+s+s1}{\PYGZsq{}}\PYG{l+s+s1}{\PYGZdl{}}\PYG{l+s+s1}{\PYGZbs{}}\PYG{l+s+s1}{psi\PYGZus{}n(x)\PYGZdl{}}\PYG{l+s+s1}{\PYGZsq{}}\PYG{p}{)}
\PYG{n}{plt}\PYG{o}{.}\PYG{n}{xlim}\PYG{p}{(}\PYG{o}{\PYGZhy{}}\PYG{l+m+mi}{10}\PYG{p}{,}\PYG{l+m+mi}{10}\PYG{p}{)}
\PYG{n}{plt}\PYG{o}{.}\PYG{n}{legend}\PYG{p}{(}\PYG{p}{)}
\PYG{n}{plt}\PYG{o}{.}\PYG{n}{show}\PYG{p}{(}\PYG{p}{)}
\end{sphinxVerbatim}

\noindent\sphinxincludegraphics{{2.62-harmonic_3_0}.png}


\chapter{Problem 2.62}
\label{\detokenize{2.62:problem-2-62}}\label{\detokenize{2.62::doc}}
\sphinxAtStartPar
Similar to problem 2.61, except now the diagonal is a function of x rather than a constant due to the varying well. Lets make our life simple and do as the book said: (\(\hbar=1, m=1, a=1\)).


\chapter{Generating the Hamilitonian}
\label{\detokenize{2.62:generating-the-hamilitonian}}
\sphinxAtStartPar
The following code will generate the Hamilitonian matrix for our problem.

\begin{sphinxVerbatim}[commandchars=\\\{\}]
\PYG{c+c1}{\PYGZsh{} Create Hamiltonian Matrix}
\PYG{k+kn}{import} \PYG{n+nn}{numpy} \PYG{k}{as} \PYG{n+nn}{np}
\PYG{k+kn}{import} \PYG{n+nn}{scipy}\PYG{n+nn}{.}\PYG{n+nn}{sparse}\PYG{n+nn}{.}\PYG{n+nn}{linalg} \PYG{k}{as} \PYG{n+nn}{sp}

\PYG{k}{def} \PYG{n+nf}{create\PYGZus{}hamiltonian}\PYG{p}{(}\PYG{n}{dimension}\PYG{p}{,} \PYG{n}{end\PYGZus{}point}\PYG{o}{=}\PYG{l+m+mi}{0}\PYG{p}{)}\PYG{p}{:}
    \PYG{c+c1}{\PYGZsh{} Let hbar = 1, m = 1.}
    \PYG{n}{ham\PYGZus{}mat} \PYG{o}{=} \PYG{n}{np}\PYG{o}{.}\PYG{n}{zeros}\PYG{p}{(}\PYG{p}{(}\PYG{n}{dimension}\PYG{p}{,}\PYG{n}{dimension}\PYG{p}{)}\PYG{p}{,} \PYG{n+nb}{float}\PYG{p}{)}

    \PYG{c+c1}{\PYGZsh{} Fill upper diagonal}
    \PYG{n}{np}\PYG{o}{.}\PYG{n}{fill\PYGZus{}diagonal}\PYG{p}{(}\PYG{n}{ham\PYGZus{}mat}\PYG{p}{[}\PYG{p}{:}\PYG{p}{,}\PYG{l+m+mi}{1}\PYG{p}{:}\PYG{p}{]}\PYG{p}{,} \PYG{o}{\PYGZhy{}}\PYG{l+m+mi}{1}\PYG{p}{)}

    \PYG{c+c1}{\PYGZsh{} Fill diagonal}
    \PYG{n}{diag} \PYG{o}{=} \PYG{p}{[}\PYG{p}{(}\PYG{l+m+mi}{2} \PYG{o}{+} \PYG{p}{(}\PYG{l+m+mi}{500} \PYG{o}{/} \PYG{p}{(}\PYG{n}{dimension}\PYG{o}{*}\PYG{o}{*}\PYG{l+m+mi}{2}\PYG{p}{)}\PYG{p}{)} \PYG{o}{*} \PYG{n}{np}\PYG{o}{.}\PYG{n}{sin}\PYG{p}{(}\PYG{n}{np}\PYG{o}{.}\PYG{n}{pi} \PYG{o}{*} \PYG{n}{j}\PYG{o}{/}\PYG{n}{dimension}\PYG{p}{)}\PYG{p}{)} \PYG{k}{for} \PYG{n}{j} \PYG{o+ow}{in} \PYG{n+nb}{range}\PYG{p}{(}\PYG{l+m+mi}{1}\PYG{p}{,}\PYG{n}{dimension}\PYG{o}{+}\PYG{l+m+mi}{1}\PYG{p}{)}\PYG{p}{]}
    \PYG{n}{np}\PYG{o}{.}\PYG{n}{fill\PYGZus{}diagonal}\PYG{p}{(}\PYG{n}{ham\PYGZus{}mat}\PYG{p}{,} \PYG{n}{diag}\PYG{p}{)}

    \PYG{c+c1}{\PYGZsh{} Fill lower diagonal}
    \PYG{n}{np}\PYG{o}{.}\PYG{n}{fill\PYGZus{}diagonal}\PYG{p}{(}\PYG{n}{ham\PYGZus{}mat}\PYG{p}{[}\PYG{l+m+mi}{1}\PYG{p}{:}\PYG{p}{]}\PYG{p}{,} \PYG{o}{\PYGZhy{}}\PYG{l+m+mi}{1}\PYG{p}{)}

    \PYG{c+c1}{\PYGZsh{} Retrieve vectors and values then return them}
    \PYG{n}{w}\PYG{p}{,} \PYG{n}{v} \PYG{o}{=} \PYG{n}{sp}\PYG{o}{.}\PYG{n}{eigs}\PYG{p}{(}\PYG{n}{ham\PYGZus{}mat}\PYG{p}{,} \PYG{n}{k}\PYG{o}{=}\PYG{l+m+mi}{3}\PYG{p}{,} \PYG{n}{which}\PYG{o}{=}\PYG{l+s+s1}{\PYGZsq{}}\PYG{l+s+s1}{SR}\PYG{l+s+s1}{\PYGZsq{}}\PYG{p}{,} \PYG{n}{return\PYGZus{}eigenvectors}\PYG{o}{=}\PYG{k+kc}{True}\PYG{p}{)}
    \PYG{k}{return} \PYG{p}{(}\PYG{n}{w}\PYG{p}{,}\PYG{n}{v}\PYG{p}{)}
\end{sphinxVerbatim}


\chapter{Using the Above Hamilitonian}
\label{\detokenize{2.62:using-the-above-hamilitonian}}
\sphinxAtStartPar
Lets use our function above to generate a 100x100 matrix and find the first three energy states along with the wave function values for these three energy states. We can then see the approximate wave function by plotting the results.

\begin{sphinxVerbatim}[commandchars=\\\{\}]
\PYG{k+kn}{import} \PYG{n+nn}{matplotlib}\PYG{n+nn}{.}\PYG{n+nn}{pyplot} \PYG{k}{as} \PYG{n+nn}{plt}
\PYG{n}{evals}\PYG{p}{,} \PYG{n}{evecs} \PYG{o}{=} \PYG{n}{create\PYGZus{}hamiltonian}\PYG{p}{(}\PYG{l+m+mi}{100}\PYG{p}{)}

\PYG{n}{plt}\PYG{o}{.}\PYG{n}{figure}\PYG{p}{(}\PYG{n}{figsize}\PYG{o}{=}\PYG{p}{(}\PYG{l+m+mi}{24}\PYG{p}{,}\PYG{l+m+mi}{8}\PYG{p}{)}\PYG{p}{)}

\PYG{c+c1}{\PYGZsh{} Plot each eigenstate}
\PYG{k}{for} \PYG{n}{n} \PYG{o+ow}{in} \PYG{n+nb}{range}\PYG{p}{(}\PYG{l+m+mi}{1}\PYG{p}{,}\PYG{l+m+mi}{4}\PYG{p}{)}\PYG{p}{:}
    \PYG{n}{plt}\PYG{o}{.}\PYG{n}{plot}\PYG{p}{(}\PYG{n}{evecs}\PYG{o}{.}\PYG{n}{real}\PYG{p}{[}\PYG{p}{:}\PYG{p}{,} \PYG{n}{n}\PYG{o}{\PYGZhy{}}\PYG{l+m+mi}{1}\PYG{p}{]}\PYG{p}{,} \PYG{n}{label}\PYG{o}{=}\PYG{l+s+sa}{fr}\PYG{l+s+s1}{\PYGZsq{}}\PYG{l+s+s1}{Energy \PYGZdl{}E\PYGZus{}}\PYG{l+s+si}{\PYGZob{}}\PYG{n}{n}\PYG{l+s+si}{\PYGZcb{}}\PYG{l+s+s1}{\PYGZdl{}: }\PYG{l+s+si}{\PYGZob{}}\PYG{l+m+mi}{100}\PYG{o}{*}\PYG{o}{*}\PYG{l+m+mi}{2}\PYG{o}{*}\PYG{n}{evals}\PYG{o}{.}\PYG{n}{real}\PYG{p}{[}\PYG{n}{n}\PYG{o}{\PYGZhy{}}\PYG{l+m+mi}{1}\PYG{p}{]}\PYG{l+s+si}{:}\PYG{l+s+s1}{.2f}\PYG{l+s+si}{\PYGZcb{}}\PYG{l+s+s1}{\PYGZsq{}}\PYG{p}{)}

\PYG{n}{plt}\PYG{o}{.}\PYG{n}{xlabel}\PYG{p}{(}\PYG{l+s+sa}{r}\PYG{l+s+s1}{\PYGZsq{}}\PYG{l+s+s1}{\PYGZdl{}x\PYGZdl{}}\PYG{l+s+s1}{\PYGZsq{}}\PYG{p}{)}
\PYG{n}{plt}\PYG{o}{.}\PYG{n}{ylabel}\PYG{p}{(}\PYG{l+s+sa}{r}\PYG{l+s+s1}{\PYGZsq{}}\PYG{l+s+s1}{\PYGZdl{}}\PYG{l+s+s1}{\PYGZbs{}}\PYG{l+s+s1}{psi\PYGZus{}n(x)\PYGZdl{}}\PYG{l+s+s1}{\PYGZsq{}}\PYG{p}{)}
\PYG{n}{plt}\PYG{o}{.}\PYG{n}{legend}\PYG{p}{(}\PYG{p}{)}
\PYG{n}{plt}\PYG{o}{.}\PYG{n}{show}\PYG{p}{(}\PYG{p}{)}
\end{sphinxVerbatim}

\noindent\sphinxincludegraphics{{2.62_3_0}.png}


\chapter{Gram\sphinxhyphen{}Schmidt Orthonormalization}
\label{\detokenize{gram-schmit:gram-schmidt-orthonormalization}}\label{\detokenize{gram-schmit::doc}}
\sphinxAtStartPar
The following produces the Gram\sphinxhyphen{}Schmit orthonormalization on arbitrary basis vectors. In the case of this example, we will be using the following vectors:

\sphinxAtStartPar
\(|𝑒1⟩=(1+𝑖)𝑖̂+(1)𝑗̂+(𝑖)𝑘̂\), \(|𝑒2⟩=(𝑖)𝑖̂+(3)𝑗̂+(1)𝑘̂\), \(|𝑒3⟩=(0)𝑖̂+(28)𝑗̂+(8)𝑘̂\)


\chapter{The Gram\sphinxhyphen{}Schmidt}
\label{\detokenize{gram-schmit:the-gram-schmidt}}
\sphinxAtStartPar
We can create our Gram\sphinxhyphen{}Schmit Orthonormalization function:

\begin{sphinxVerbatim}[commandchars=\\\{\}]
\PYG{k+kn}{import} \PYG{n+nn}{numpy} \PYG{k}{as} \PYG{n+nn}{np}
\PYG{k}{def} \PYG{n+nf}{gram\PYGZus{}schmidt}\PYG{p}{(}\PYG{n}{vectors}\PYG{p}{)}\PYG{p}{:}
    \PYG{n}{bases} \PYG{o}{=} \PYG{p}{[}\PYG{p}{]}
    \PYG{k}{for} \PYG{n}{vec} \PYG{o+ow}{in} \PYG{n}{vectors}\PYG{p}{:}
        \PYG{n}{w} \PYG{o}{=} \PYG{n}{vec} \PYG{o}{\PYGZhy{}} \PYG{n}{np}\PYG{o}{.}\PYG{n}{sum}\PYG{p}{(}\PYG{n}{np}\PYG{o}{.}\PYG{n}{dot}\PYG{p}{(}\PYG{n}{vec}\PYG{p}{,} \PYG{n}{q}\PYG{p}{)} \PYG{o}{*} \PYG{n}{q} \PYG{k}{for} \PYG{n}{q} \PYG{o+ow}{in} \PYG{n}{bases}\PYG{p}{)}
        \PYG{n}{bases}\PYG{o}{.}\PYG{n}{append}\PYG{p}{(}\PYG{n}{w} \PYG{o}{/} \PYG{n}{np}\PYG{o}{.}\PYG{n}{linalg}\PYG{o}{.}\PYG{n}{norm}\PYG{p}{(}\PYG{n}{w}\PYG{p}{)}\PYG{p}{)}
    \PYG{k}{return} \PYG{n}{np}\PYG{o}{.}\PYG{n}{array}\PYG{p}{(}\PYG{n}{bases}\PYG{p}{)}
\end{sphinxVerbatim}


\chapter{Example Bases}
\label{\detokenize{gram-schmit:example-bases}}
\sphinxAtStartPar
Let’s now use our \sphinxcode{\sphinxupquote{gram\_schmidt}} function to evaluate the original bases given at the top of this page:

\sphinxAtStartPar
\(|𝑒1⟩=(1+𝑖)𝑖̂+(1)𝑗̂+(𝑖)𝑘̂\), \(|𝑒2⟩=(𝑖)𝑖̂+(3)𝑗̂+(1)𝑘̂\), \(|𝑒3⟩=(0)𝑖̂+(28)𝑗̂+(8)𝑘̂\)

\begin{sphinxVerbatim}[commandchars=\\\{\}]
\PYG{n}{bases} \PYG{o}{=} \PYG{n}{np}\PYG{o}{.}\PYG{n}{array}\PYG{p}{(}\PYG{p}{[}
    \PYG{p}{[}\PYG{l+m+mi}{1}\PYG{o}{+}\PYG{l+m+mf}{1.}\PYG{n}{j}\PYG{p}{,} \PYG{l+m+mf}{1.}\PYG{p}{,} \PYG{l+m+mf}{1.}\PYG{n}{j}\PYG{p}{]}\PYG{p}{,}
    \PYG{p}{[}\PYG{l+m+mf}{1.}\PYG{n}{j}\PYG{p}{,} \PYG{l+m+mf}{3.}\PYG{p}{,} \PYG{l+m+mf}{1.}\PYG{p}{]}\PYG{p}{,}
    \PYG{p}{[}\PYG{l+m+mf}{0.}\PYG{p}{,} \PYG{l+m+mf}{28.}\PYG{p}{,} \PYG{l+m+mf}{8.}\PYG{p}{]}
\PYG{p}{]}\PYG{p}{)}\PYG{o}{.}\PYG{n}{T}

\PYG{n+nb}{print}\PYG{p}{(}\PYG{n}{gram\PYGZus{}schmidt}\PYG{p}{(}\PYG{n}{bases}\PYG{p}{)}\PYG{p}{)}
\end{sphinxVerbatim}

\begin{sphinxVerbatim}[commandchars=\\\{\}]
[[0.57735027+0.57735027j 0.        +0.57735027j 0.        +0.j        ]
 [0.0702873 \PYGZhy{}0.05857275j 0.15228915\PYGZhy{}0.01171455j 0.98402221+0.j        ]
 [0.31547059+0.83647199j 0.32032088+0.30859935j 0.03395199\PYGZhy{}0.04243999j]]
\end{sphinxVerbatim}

\begin{sphinxVerbatim}[commandchars=\\\{\}]
\PYGZlt{}ipython\PYGZhy{}input\PYGZhy{}1\PYGZhy{}fb9ee8129d3d\PYGZgt{}:5: DeprecationWarning: Calling np.sum(generator) is deprecated, and in the future will give a different result. Use np.sum(np.fromiter(generator)) or the python sum builtin instead.
  w = vec \PYGZhy{} np.sum(np.dot(vec, q) * q for q in bases)
\end{sphinxVerbatim}


\chapter{Another Example}
\label{\detokenize{gram-schmit:another-example}}
\sphinxAtStartPar
Below I test the following basis vectors as required by the Homework 3B assignment:
\( |𝑒1⟩=(1+𝑖)𝑖̂+(1)𝑗̂+(𝑖)𝑘̂, |𝑒2⟩=(𝑖)𝑖̂+(3)𝑗̂+(1)𝑘̂, |𝑒3⟩=(0)𝑖̂+(28)𝑗̂+(8)𝑘\)

\sphinxAtStartPar
Note: I enter them as row vectors, then transpose them using .T as it’s easier to type.

\begin{sphinxVerbatim}[commandchars=\\\{\}]
\PYG{n}{bases\PYGZus{}2} \PYG{o}{=} \PYG{n}{np}\PYG{o}{.}\PYG{n}{array}\PYG{p}{(}\PYG{p}{[}
    \PYG{p}{[}\PYG{l+m+mi}{1}\PYG{o}{+}\PYG{l+m+mf}{1.}\PYG{n}{j}\PYG{p}{,} \PYG{l+m+mi}{1}\PYG{p}{,} \PYG{l+m+mf}{1.}\PYG{n}{j}\PYG{p}{]}\PYG{p}{,}
    \PYG{p}{[}\PYG{l+m+mf}{1.}\PYG{n}{j}\PYG{p}{,} \PYG{l+m+mi}{3}\PYG{p}{,} \PYG{l+m+mi}{1}\PYG{p}{]}\PYG{p}{,}
    \PYG{p}{[}\PYG{l+m+mi}{0}\PYG{p}{,} \PYG{l+m+mi}{28}\PYG{p}{,} \PYG{l+m+mi}{8}\PYG{p}{]}
\PYG{p}{]}\PYG{p}{)}\PYG{o}{.}\PYG{n}{T}

\PYG{n+nb}{print}\PYG{p}{(}\PYG{n}{gram\PYGZus{}schmidt}\PYG{p}{(}\PYG{n}{bases\PYGZus{}2}\PYG{p}{)}\PYG{p}{)}
\end{sphinxVerbatim}

\begin{sphinxVerbatim}[commandchars=\\\{\}]
[[0.57735027+0.57735027j 0.        +0.57735027j 0.        +0.j        ]
 [0.0702873 \PYGZhy{}0.05857275j 0.15228915\PYGZhy{}0.01171455j 0.98402221+0.j        ]
 [0.31547059+0.83647199j 0.32032088+0.30859935j 0.03395199\PYGZhy{}0.04243999j]]
\end{sphinxVerbatim}

\begin{sphinxVerbatim}[commandchars=\\\{\}]
\PYGZlt{}ipython\PYGZhy{}input\PYGZhy{}1\PYGZhy{}fb9ee8129d3d\PYGZgt{}:5: DeprecationWarning: Calling np.sum(generator) is deprecated, and in the future will give a different result. Use np.sum(np.fromiter(generator)) or the python sum builtin instead.
  w = vec \PYGZhy{} np.sum(np.dot(vec, q) * q for q in bases)
\end{sphinxVerbatim}


\chapter{Finding Bohr Energies in Hydrogen}
\label{\detokenize{4.69:finding-bohr-energies-in-hydrogen}}\label{\detokenize{4.69::doc}}
\sphinxAtStartPar
We can use the ‘wag the dog’ method to find Bohr energies in the Hydrogen atom. From the book, we know that
\begin{equation*}
\begin{split}
\frac{d^2u}{d\rho^2} = \left[ 1-\frac{\rho_0}{\rho}+\frac{l(l+1)}{\rho^2} \right] u
\end{split}
\end{equation*}
\sphinxAtStartPar
Let’s let
\begin{equation*}
\begin{split}
\rho_0 = 2n
\end{split}
\end{equation*}
\sphinxAtStartPar
Then we can find the discrete energy levels using the ‘wag the dog’ method. We’ll do it for 3 discrete \(n\) for each \(l \in \{0,1,2\}\)

\begin{sphinxVerbatim}[commandchars=\\\{\}]
\PYG{k+kn}{import} \PYG{n+nn}{numpy} \PYG{k}{as} \PYG{n+nn}{np}
\PYG{k+kn}{from} \PYG{n+nn}{scipy}\PYG{n+nn}{.}\PYG{n+nn}{integrate} \PYG{k+kn}{import} \PYG{n}{solve\PYGZus{}ivp}
\PYG{k+kn}{import} \PYG{n+nn}{matplotlib}\PYG{n+nn}{.}\PYG{n+nn}{pyplot} \PYG{k}{as} \PYG{n+nn}{plt}

\PYG{k}{def} \PYG{n+nf}{wag\PYGZus{}the\PYGZus{}dog}\PYG{p}{(}\PYG{n}{x\PYGZus{}range}\PYG{o}{=}\PYG{p}{(}\PYG{l+m+mf}{0.0000001}\PYG{p}{,}\PYG{l+m+mi}{25}\PYG{p}{)}\PYG{p}{,} \PYG{n}{n\PYGZus{}values}\PYG{o}{=}\PYG{n}{np}\PYG{o}{.}\PYG{n}{linspace}\PYG{p}{(}\PYG{l+m+mf}{0.9}\PYG{p}{,} \PYG{l+m+mf}{1.1}\PYG{p}{,} \PYG{l+m+mi}{10}\PYG{p}{)}\PYG{p}{,} \PYG{n}{initial\PYGZus{}values}\PYG{o}{=}\PYG{p}{(}\PYG{l+m+mf}{1.}\PYG{p}{,}\PYG{l+m+mf}{0.}\PYG{p}{)}\PYG{p}{,} \PYG{n}{l}\PYG{o}{=}\PYG{l+m+mi}{0}\PYG{p}{)}\PYG{p}{:}
    \PYG{n}{positions} \PYG{o}{=} \PYG{n}{np}\PYG{o}{.}\PYG{n}{linspace}\PYG{p}{(}\PYG{n}{x\PYGZus{}range}\PYG{p}{[}\PYG{l+m+mi}{0}\PYG{p}{]}\PYG{p}{,} \PYG{n}{x\PYGZus{}range}\PYG{p}{[}\PYG{l+m+mi}{1}\PYG{p}{]}\PYG{p}{,} \PYG{l+m+mi}{1000}\PYG{p}{)}
    \PYG{n}{plt}\PYG{o}{.}\PYG{n}{figure}\PYG{p}{(}\PYG{n}{figsize}\PYG{o}{=}\PYG{p}{(}\PYG{l+m+mi}{24}\PYG{p}{,}\PYG{l+m+mi}{8}\PYG{p}{)}\PYG{p}{)}

    \PYG{k}{for} \PYG{n}{n\PYGZus{}value} \PYG{o+ow}{in} \PYG{n}{n\PYGZus{}values}\PYG{p}{:}  
        \PYG{c+c1}{\PYGZsh{} Differential equation}
        \PYG{n}{psi\PYGZus{}prime} \PYG{o}{=} \PYG{k}{lambda} \PYG{n}{rho}\PYG{p}{,} \PYG{n}{psi}\PYG{p}{,} \PYG{n}{n}\PYG{p}{:} \PYG{p}{[}\PYG{n}{psi}\PYG{p}{[}\PYG{l+m+mi}{1}\PYG{p}{]}\PYG{p}{,} \PYG{p}{(}\PYG{l+m+mi}{1}\PYG{o}{\PYGZhy{}}\PYG{p}{(}\PYG{l+m+mi}{2}\PYG{o}{*}\PYG{n}{n}\PYG{o}{/}\PYG{p}{(}\PYG{n}{rho}\PYG{p}{)}\PYG{p}{)} \PYG{o}{+} \PYG{p}{(}\PYG{n}{l}\PYG{o}{*}\PYG{p}{(}\PYG{n}{l}\PYG{o}{+}\PYG{l+m+mi}{1}\PYG{p}{)}\PYG{o}{/}\PYG{p}{(}\PYG{n}{rho}\PYG{o}{*}\PYG{o}{*}\PYG{l+m+mi}{2}\PYG{p}{)}\PYG{p}{)}\PYG{p}{)}\PYG{o}{*}\PYG{n}{psi}\PYG{p}{[}\PYG{l+m+mi}{0}\PYG{p}{]}\PYG{p}{]}

        \PYG{c+c1}{\PYGZsh{} Solve differential equation using scipy}
        \PYG{n}{sol} \PYG{o}{=} \PYG{n}{solve\PYGZus{}ivp}\PYG{p}{(}\PYG{n}{psi\PYGZus{}prime}\PYG{p}{,} \PYG{n}{x\PYGZus{}range}\PYG{p}{,} \PYG{n}{initial\PYGZus{}values}\PYG{p}{,} \PYG{n}{t\PYGZus{}eval}\PYG{o}{=}\PYG{n}{positions}\PYG{p}{,} \PYG{n}{args}\PYG{o}{=}\PYG{p}{[}\PYG{n}{n\PYGZus{}value}\PYG{p}{]}\PYG{p}{)}

        \PYG{c+c1}{\PYGZsh{} Plot solution}
        \PYG{n}{plt}\PYG{o}{.}\PYG{n}{plot}\PYG{p}{(}\PYG{n}{positions}\PYG{p}{,} \PYG{n}{sol}\PYG{o}{.}\PYG{n}{y}\PYG{p}{[}\PYG{l+m+mi}{0}\PYG{p}{]}\PYG{p}{,} \PYG{n}{label}\PYG{o}{=}\PYG{l+s+sa}{fr}\PYG{l+s+s1}{\PYGZsq{}}\PYG{l+s+s1}{\PYGZdl{}n\PYGZdl{}: }\PYG{l+s+si}{\PYGZob{}}\PYG{n}{n\PYGZus{}value}\PYG{l+s+si}{:}\PYG{l+s+s1}{.4f}\PYG{l+s+si}{\PYGZcb{}}\PYG{l+s+s1}{, \PYGZdl{}l\PYGZdl{}: }\PYG{l+s+si}{\PYGZob{}}\PYG{n}{l}\PYG{l+s+si}{\PYGZcb{}}\PYG{l+s+s1}{\PYGZsq{}}\PYG{p}{)}

    \PYG{c+c1}{\PYGZsh{} Plotting configuration}
    \PYG{n}{plt}\PYG{o}{.}\PYG{n}{legend}\PYG{p}{(}\PYG{p}{)}
    \PYG{n}{plt}\PYG{o}{.}\PYG{n}{axhline}\PYG{p}{(}\PYG{n}{c}\PYG{o}{=}\PYG{l+s+s1}{\PYGZsq{}}\PYG{l+s+s1}{black}\PYG{l+s+s1}{\PYGZsq{}}\PYG{p}{)}
    \PYG{n}{plt}\PYG{o}{.}\PYG{n}{xlabel}\PYG{p}{(}\PYG{l+s+sa}{r}\PYG{l+s+s1}{\PYGZsq{}}\PYG{l+s+s1}{\PYGZdl{}}\PYG{l+s+s1}{\PYGZbs{}}\PYG{l+s+s1}{rho\PYGZdl{}}\PYG{l+s+s1}{\PYGZsq{}}\PYG{p}{)}
    \PYG{n}{plt}\PYG{o}{.}\PYG{n}{ylabel}\PYG{p}{(}\PYG{l+s+sa}{r}\PYG{l+s+s1}{\PYGZsq{}}\PYG{l+s+s1}{\PYGZdl{}u\PYGZus{}n(}\PYG{l+s+s1}{\PYGZbs{}}\PYG{l+s+s1}{rho)\PYGZdl{}}\PYG{l+s+s1}{\PYGZsq{}}\PYG{p}{)}
    \PYG{n}{plt}\PYG{o}{.}\PYG{n}{ylim}\PYG{p}{(}\PYG{o}{\PYGZhy{}}\PYG{l+m+mi}{5}\PYG{p}{,}\PYG{l+m+mi}{5}\PYG{p}{)}
    \PYG{n}{plt}\PYG{o}{.}\PYG{n}{show}\PYG{p}{(}\PYG{p}{)}
\end{sphinxVerbatim}


\chapter{Visualization}
\label{\detokenize{4.69:visualization}}
\sphinxAtStartPar
for \(l=0\):

\begin{sphinxVerbatim}[commandchars=\\\{\}]
\PYG{n}{wag\PYGZus{}the\PYGZus{}dog}\PYG{p}{(}\PYG{n}{initial\PYGZus{}values}\PYG{o}{=}\PYG{p}{(}\PYG{l+m+mf}{0.}\PYG{p}{,} \PYG{l+m+mf}{1.}\PYG{p}{)}\PYG{p}{,} \PYG{n}{n\PYGZus{}values}\PYG{o}{=}\PYG{p}{[}\PYG{l+m+mf}{0.9999}\PYG{p}{,} \PYG{l+m+mf}{1.0001}\PYG{p}{,} \PYG{l+m+mf}{1.9999}\PYG{p}{,} \PYG{l+m+mf}{2.0001}\PYG{p}{,} \PYG{l+m+mf}{2.9998}\PYG{p}{,} \PYG{l+m+mf}{3.0001}\PYG{p}{]}\PYG{p}{,} \PYG{n}{l}\PYG{o}{=}\PYG{l+m+mi}{0}\PYG{p}{)}
\end{sphinxVerbatim}

\noindent\sphinxincludegraphics{{4.69_4_0}.png}

\sphinxAtStartPar
for \(l=1\), our initial value problem becomes a boundary value problem. This is due to \(u(1)=1, u'(0)=0\)

\begin{sphinxVerbatim}[commandchars=\\\{\}]
\PYG{k+kn}{from} \PYG{n+nn}{scipy}\PYG{n+nn}{.}\PYG{n+nn}{integrate} \PYG{k+kn}{import} \PYG{n}{solve\PYGZus{}bvp}
\PYG{k}{def} \PYG{n+nf}{wag\PYGZus{}the\PYGZus{}dog2}\PYG{p}{(}\PYG{n}{x\PYGZus{}range}\PYG{o}{=}\PYG{p}{(}\PYG{l+m+mf}{0.0000001}\PYG{p}{,}\PYG{l+m+mi}{25}\PYG{p}{)}\PYG{p}{,} \PYG{n}{n\PYGZus{}values}\PYG{o}{=}\PYG{n}{np}\PYG{o}{.}\PYG{n}{linspace}\PYG{p}{(}\PYG{l+m+mf}{0.9}\PYG{p}{,} \PYG{l+m+mf}{1.1}\PYG{p}{,} \PYG{l+m+mi}{10}\PYG{p}{)}\PYG{p}{,} \PYG{n}{initial\PYGZus{}u}\PYG{o}{=}\PYG{l+m+mf}{1.}\PYG{p}{,} \PYG{n}{initial\PYGZus{}up}\PYG{o}{=}\PYG{l+m+mf}{0.}\PYG{p}{,} \PYG{n}{l}\PYG{o}{=}\PYG{l+m+mi}{0}\PYG{p}{)}\PYG{p}{:}
    \PYG{n}{positions} \PYG{o}{=} \PYG{n}{np}\PYG{o}{.}\PYG{n}{linspace}\PYG{p}{(}\PYG{n}{x\PYGZus{}range}\PYG{p}{[}\PYG{l+m+mi}{0}\PYG{p}{]}\PYG{p}{,} \PYG{n}{x\PYGZus{}range}\PYG{p}{[}\PYG{l+m+mi}{1}\PYG{p}{]}\PYG{p}{,} \PYG{l+m+mi}{1000}\PYG{p}{)}
    \PYG{n}{plt}\PYG{o}{.}\PYG{n}{figure}\PYG{p}{(}\PYG{n}{figsize}\PYG{o}{=}\PYG{p}{(}\PYG{l+m+mi}{24}\PYG{p}{,}\PYG{l+m+mi}{8}\PYG{p}{)}\PYG{p}{)}

    \PYG{k}{for} \PYG{n}{n\PYGZus{}value} \PYG{o+ow}{in} \PYG{n}{n\PYGZus{}values}\PYG{p}{:}  
        \PYG{c+c1}{\PYGZsh{} Differential equation}
        \PYG{n}{u\PYGZus{}pp} \PYG{o}{=} \PYG{k}{lambda} \PYG{n}{rho}\PYG{p}{,} \PYG{n}{u}\PYG{p}{:} \PYG{p}{[}\PYG{n}{u}\PYG{p}{[}\PYG{l+m+mi}{1}\PYG{p}{]}\PYG{p}{,} \PYG{p}{(}\PYG{l+m+mi}{1} \PYG{o}{\PYGZhy{}} \PYG{p}{(}\PYG{l+m+mi}{2}\PYG{o}{*}\PYG{n}{n\PYGZus{}value}\PYG{o}{/}\PYG{p}{(}\PYG{n}{rho}\PYG{p}{)}\PYG{p}{)} \PYG{o}{+} \PYG{p}{(}\PYG{n}{l}\PYG{o}{*}\PYG{p}{(}\PYG{n}{l}\PYG{o}{+}\PYG{l+m+mi}{1}\PYG{p}{)}\PYG{o}{/}\PYG{p}{(}\PYG{n}{rho}\PYG{o}{*}\PYG{o}{*}\PYG{l+m+mi}{2}\PYG{p}{)}\PYG{p}{)}\PYG{p}{)}\PYG{o}{*}\PYG{n}{u}\PYG{p}{[}\PYG{l+m+mi}{0}\PYG{p}{]}\PYG{p}{]}
        \PYG{n}{bc}   \PYG{o}{=} \PYG{k}{lambda} \PYG{n}{u1}\PYG{p}{,} \PYG{n}{u2}\PYG{p}{:} \PYG{p}{[}\PYG{n}{u1}\PYG{p}{[}\PYG{l+m+mi}{0}\PYG{p}{]}\PYG{o}{\PYGZhy{}}\PYG{n}{initial\PYGZus{}u}\PYG{p}{,} \PYG{n}{u2}\PYG{p}{[}\PYG{l+m+mi}{0}\PYG{p}{]}\PYG{o}{\PYGZhy{}}\PYG{n}{initial\PYGZus{}up}\PYG{p}{]}

        \PYG{n}{sol} \PYG{o}{=} \PYG{n}{solve\PYGZus{}bvp}\PYG{p}{(}\PYG{n}{u\PYGZus{}pp}\PYG{p}{,} \PYG{n}{bc}\PYG{p}{,} \PYG{p}{[}\PYG{l+m+mf}{0.0000001}\PYG{p}{,} \PYG{l+m+mf}{1.}\PYG{p}{]}\PYG{p}{,} \PYG{p}{[}\PYG{p}{[}\PYG{l+m+mf}{0.}\PYG{p}{,}\PYG{l+m+mf}{0.}\PYG{p}{]}\PYG{p}{,}\PYG{p}{[}\PYG{l+m+mf}{1.}\PYG{p}{,}\PYG{l+m+mf}{0.}\PYG{p}{]}\PYG{p}{]}\PYG{p}{)}

        \PYG{c+c1}{\PYGZsh{} Plot solution}
        \PYG{n}{plt}\PYG{o}{.}\PYG{n}{plot}\PYG{p}{(}\PYG{n}{sol}\PYG{o}{.}\PYG{n}{x}\PYG{p}{,} \PYG{n}{sol}\PYG{o}{.}\PYG{n}{y}\PYG{p}{[}\PYG{l+m+mi}{0}\PYG{p}{]}\PYG{p}{,} \PYG{n}{label}\PYG{o}{=}\PYG{l+s+sa}{fr}\PYG{l+s+s1}{\PYGZsq{}}\PYG{l+s+s1}{\PYGZdl{}n\PYGZdl{}: }\PYG{l+s+si}{\PYGZob{}}\PYG{n}{n\PYGZus{}value}\PYG{l+s+si}{:}\PYG{l+s+s1}{.4f}\PYG{l+s+si}{\PYGZcb{}}\PYG{l+s+s1}{, \PYGZdl{}l\PYGZdl{}: }\PYG{l+s+si}{\PYGZob{}}\PYG{n}{l}\PYG{l+s+si}{\PYGZcb{}}\PYG{l+s+s1}{\PYGZsq{}}\PYG{p}{)}

    \PYG{c+c1}{\PYGZsh{} Plotting configuration}
    \PYG{n}{plt}\PYG{o}{.}\PYG{n}{legend}\PYG{p}{(}\PYG{p}{)}
    \PYG{n}{plt}\PYG{o}{.}\PYG{n}{axhline}\PYG{p}{(}\PYG{n}{c}\PYG{o}{=}\PYG{l+s+s1}{\PYGZsq{}}\PYG{l+s+s1}{black}\PYG{l+s+s1}{\PYGZsq{}}\PYG{p}{)}
    \PYG{n}{plt}\PYG{o}{.}\PYG{n}{xlabel}\PYG{p}{(}\PYG{l+s+sa}{r}\PYG{l+s+s1}{\PYGZsq{}}\PYG{l+s+s1}{\PYGZdl{}}\PYG{l+s+s1}{\PYGZbs{}}\PYG{l+s+s1}{rho\PYGZdl{}}\PYG{l+s+s1}{\PYGZsq{}}\PYG{p}{)}
    \PYG{n}{plt}\PYG{o}{.}\PYG{n}{ylabel}\PYG{p}{(}\PYG{l+s+sa}{r}\PYG{l+s+s1}{\PYGZsq{}}\PYG{l+s+s1}{\PYGZdl{}u\PYGZus{}n(}\PYG{l+s+s1}{\PYGZbs{}}\PYG{l+s+s1}{rho)\PYGZdl{}}\PYG{l+s+s1}{\PYGZsq{}}\PYG{p}{)}
    \PYG{n}{plt}\PYG{o}{.}\PYG{n}{ylim}\PYG{p}{(}\PYG{o}{\PYGZhy{}}\PYG{l+m+mi}{5}\PYG{p}{,}\PYG{l+m+mi}{5}\PYG{p}{)}
    \PYG{n}{plt}\PYG{o}{.}\PYG{n}{show}\PYG{p}{(}\PYG{p}{)}

\PYG{n}{wag\PYGZus{}the\PYGZus{}dog2}\PYG{p}{(}\PYG{n}{x\PYGZus{}range}\PYG{o}{=}\PYG{p}{(}\PYG{l+m+mi}{1}\PYG{p}{,}\PYG{l+m+mi}{25}\PYG{p}{)}\PYG{p}{,} \PYG{n}{n\PYGZus{}values}\PYG{o}{=}\PYG{p}{[}\PYG{l+m+mf}{0.9999}\PYG{p}{,} \PYG{l+m+mf}{1.0001}\PYG{p}{,} \PYG{l+m+mf}{1.9999}\PYG{p}{,} \PYG{l+m+mf}{2.0001}\PYG{p}{,} \PYG{l+m+mf}{2.9998}\PYG{p}{,} \PYG{l+m+mf}{3.0001}\PYG{p}{]}\PYG{p}{,} \PYG{n}{l}\PYG{o}{=}\PYG{l+m+mi}{2}\PYG{p}{)}
\end{sphinxVerbatim}

\noindent\sphinxincludegraphics{{4.69_6_0}.png}

\sphinxAtStartPar
\(l=2\)

\begin{sphinxVerbatim}[commandchars=\\\{\}]
\PYG{n}{wag\PYGZus{}the\PYGZus{}dog2}\PYG{p}{(}\PYG{n}{x\PYGZus{}range}\PYG{o}{=}\PYG{p}{(}\PYG{l+m+mi}{1}\PYG{p}{,}\PYG{l+m+mi}{25}\PYG{p}{)}\PYG{p}{,} \PYG{n}{n\PYGZus{}values}\PYG{o}{=}\PYG{p}{[}\PYG{l+m+mf}{0.9999}\PYG{p}{,} \PYG{l+m+mf}{1.0001}\PYG{p}{,} \PYG{l+m+mf}{1.9999}\PYG{p}{,} \PYG{l+m+mf}{2.0001}\PYG{p}{,} \PYG{l+m+mf}{2.9998}\PYG{p}{,} \PYG{l+m+mf}{3.0001}\PYG{p}{]}\PYG{p}{,} \PYG{n}{l}\PYG{o}{=}\PYG{l+m+mi}{3}\PYG{p}{)}
\end{sphinxVerbatim}

\noindent\sphinxincludegraphics{{4.69_8_0}.png}







\renewcommand{\indexname}{Index}
\printindex
\end{document}